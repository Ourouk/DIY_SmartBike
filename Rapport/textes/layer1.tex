\chapter{Hardware}
\section{Raspberry Pi}
L'intégration d'une Raspberry Pi dans l'écosystème de l'Internet des Objets (IoT) est une démarche pragmatique et fonctionnelle. Avec la version récente de Raspbian, baptisée Bullseye, l'accent est mis sur la stabilité et la compatibilité, ce qui en fait un choix solide pour les projets IoT. En utilisant cette configuration, il est possible de mettre en place des dispositifs qui collectent des données provenant de capteurs divers et les transmettent à des systèmes centraux pour analyse.
Nous combinons les capacité de la rpi avec Node-red,  en tant qu'outil de programmation visuelle, facilite la création de flux de données et d'automatisations en utilisant des nœuds interconnectés (nous y revienderons plus tard).
\subsection{GrovePi}
GrovePi, une extension matérielle conçue pour la Raspberry Pi, constitue une solution intéressante pour étendre les capacités de la carte dans le domaine de l'Internet des Objets (IoT). En intégrant des connecteurs Grove standardisés, GrovePi simplifie la connexion de divers capteurs et actionneurs sans nécessiter de soudure ou de câblage complexe.

Nous avons utilisé les bibliothèques standarts fournies par le projets, il est à noté que nous avons du faire de petite modifications, pour rendre compatible grovePI et Bulleyes.
\subsection{LORA HAT}
\section{ESP32}
L'ESP32 intègre un processeur à deux cœurs, une connectivité Wi-Fi et Bluetooth, ainsi qu'une variété d'interfaces périphériques. Grâce à sa puissance de traitement, il peut exécuter des tâches complexes et prendre en charge des fonctionnalités avancées. En tant que plate-forme de développement, l'ESP32 est pris en charge par diverses bibliothèques et outils de programmation, ce qui en fait un choix polyvalent pour les développeurs.

Il recoit des données du GPS (via les gpip)et les envoies via LORA à notre RPI.
\section{ESP8266}
Ce module est doté d'un microcontrôleur intégré avec une connectivité Wi-Fi intégrée. Bien que moins puissant que l'ESP32, l'ESP8266 reste tout à fait capable d'exécuter des tâches IoT courantes telles que la collecte et la transmission de données. Son faible coût et sa faible consommation d'énergie en font un choix populaire pour les projets nécessitant une connectivité Wi-Fi.
Dans le cadre de notre projet sont utilisations reste très basique, mais nous avons eu l'occasion d'entrevoir son grand potentiel.  Simple activation de led.


\section{estimation de prix}

\begin{itemize}
	\item raspberri pi 
	
	€150
	\item LORA hat
	
	€17 - €42
	\item Pi cam
	
	€9 - €25
	\item Light sensor
	
	€1.70 - €8.50
	\item RFID
	
	€4 - €17
	\item Button
	
	€0.85 - €4.20
	\item LCD screen
	
	€8.50 - €25.50
	\item LED
	
	€0.08 - €0.85
	\item Buzzer
	
	€0.85 - €4.20
	\item Relay
	
	€1.70 - €8.50
	\item Gyroscope
	
	€4 - €17
	
	\item ESP32	
	€5 - €15
	
	\item gps
	
	€10 - €40
	
	\item ESP8266	
	€5 - €10
	
	\item température sensor	
 
	€2 - €10
\end{itemize}


\chapter{Le marché }

le prix total du produit se trouve  €170 - €380.Il faut bien se rendre compte que le produit est une version de test pour utiliser le maximum de technologie IOT. Dans une application plus réelle de notre solution. nous utiliserons une seule carte pour centraliser les mesures. Donc il est facilement réalisable d'économiser une 100 ene d'euros.et en choisissant de bons composants pas trop chère en grande quantité le cout final du produit ne sera pas plus de 70 euros. Il faut noter que pour réaliser un tel gain, il est nécessaire de recourir à l'ingénierie électronique en fabriquant une carte mère dédier. en fixant cet objectif le produit pourrait se vendre sans trop de problème pour €200 tout en pouvant dégager une marche agréable afin de rembourser le R \& D , le marketing et les diverses charges…  


\section{Étude de marché }


Segmentation de la Clientèle : Notre produit cible principalement les centres urbains où la demande de solutions de mobilité alternatives est élevée. Les jeunes professionnels, les étudiants et les citadins soucieux de l'environnement sont les segments clés visés.


Avantages Concurrentiels : En offrant des fonctionnalités telles que la détection d'accident, la sécurité renforcée et la connectivité GPS à un prix abordable, notre vélo intelligent se démarque de la concurrence. Cela attire des clients cherchant un équilibre entre fonctionnalités haut de gamme et budget.


Potentiel de Croissance : Avec des prévisions indiquant une augmentation continue de la demande de vélos intelligents et d'accessoires, notre produit abordable a le potentiel de capturer une part substantielle de ce marché en pleine croissance.

Notre proposition de vélo intelligent abordable trouve sa place dans l'évolution constante du cyclisme en Belgique. Au cours des trois dernières années, une moyenne de 580 000 vélos a été vendue, dont 2/5 sont électriques, témoignant d'un intérêt grandissant pour des solutions de mobilité modernes. Avec une pénétration du marché estimée à 1\%, cela ouvre la voie à la vente de 5 800 appareils par an uniquement en Belgique. Évalué à 200 euros par unité, ce potentiel se traduit par un chiffre d'affaires de 1 million d'euros par ans. En ciblant initialement le marché belge, notre produit aspire à capitaliser sur cette opportunité naissante et à répondre à la demande croissante en matière de vélos intelligents abordables.

