\chapter{Cadrage Théorique}
\section{Introduction}

La consommation de drogues au sein de la population estudiantine représente un problème important.  Cette pratique est en augmentation et devient de plus en plus fréquente. En effet, selon Louisa Degenhardt et Al, il y avait 34 millions de consommateurs d'amphétamine et 17 millions de consommateurs de cocaïne dans le monde en 2010. 17,2 millions étaient des consommateurs dépendants d'amphétamine et 6,9 millions étaient des consommateurs dépendants de cocaïne \parencite{degenhardt_global_2014}. \newline


Selon Johan Duflou, le nombre de consommateurs d'amphétamine est resté relativement stable dans les 6 années qui ont suivi alors que pour les consommateurs de cocaïne, il y a eu une augmentation de 7\% \parencite{duflou_psychostimulant_2020}. \newline

Selon Safia Sharif et Al, la prévalence d'utilisation abusive des psychostimulants par les étudiants universitaires dans le monde se situe entre 6 et 20\%. Ces données ont été récoltées aux États-Unis, au Royaume-Uni, en Australie, en France et en Italie \parencite{sharif_use_2021}. \newline

D'après une étude réalisée en Belgique en 2020, l'âge médian auquel seraient initiés les jeunes  à l'usage non médical des médicaments est de 18 ans pour les stimulants \parencite{c_guillain_les_2021}. 
Aux États-Unis, le mésusage de stimulants est plus fréquent dans la tranche d'âge 12-17 ans que chez les plus âgés et le groupe ethnique le plus touché est celui des caucasiens. Cela représente 4\% des étudiants du secondaire. Une étude a montré que parmi les étudiants ayant recours à cette pratique dans le cycle supérieur, 21\% avaient débuté à l'école secondaire. Il y aurait donc un phénomène de continuité assez important \parencite{levy_usages_2008-1}.




\section{Définition}

Un psychostimulant ou nootrope est "une substance de type supplément alimentaire, médicament, drogue, plante ou autre permettant de stimuler la conscience, la pensée et les capacités cognitives" \parencite{noauthor_nootrope_nodate, suliman_establishing_2016}. \newline
L'usage non médical d'un médicament est le "fait d'utiliser un médicament psychoactif à des fins d'automédication, divertissement ou dans le but d'améliorer ses performances, en dehors des directives médicales acceptées" \parencite{european_monitoring_centre_for_drugs_and_drug_addiction_usage_2021}. \newline

Le principe de cette consommation est donc le dopage cognitif. Les étudiants y ayant recours cherchent à rester éveillé plus longtemps, améliorer la mémoire, créativité et concentration pour un meilleur apprentissage, augmenter la motivation, améliorer leur condition de vie (mieux gérer le stress, combattre l'anxiété, améliorer le sommeil, contrôler la fatigue mentale ainsi que l'humeur), faciliter la communication lors d'un examen oral ou tout simplement à se détendre. Cependant, c'est une forme de dopage à différents degrés : léger (thé, café, etc.), moyen (boissons énergisantes) ou extrême (amphétaminiques, cocaïne, etc.). Le type de consommation peut donc avoir des effets et conséquences différents \parencite{carton_dopage_2018}. \newline 



Beaucoup d'auteurs ont répertorié différentes substances consommées par les étudiants. 
On peut y retrouver : 

\begin{itemize}
\item
  Certains médicaments sur ordonnance (anxiolytiques/tranquillisants,
  somnifères, amphétaminiques, antidépresseurs, \ldots)
\item
  Certaines substances illicites (cocaïne, ecstasy, cannabis, \ldots)
\item
  Les molécules de synthèse pas encore interdites (2-Phényléthylamine,
  1p-LSD, \ldots)
\item
  Ainsi que les produits en vente libre comprenant les boissons
  énergisantes, les compléments alimentaires stimulants (comprimés de
  caféine, extrait de ginkgo biloba, ginseng, \ldots) le tabac et
  l'alcool\parencite{shakeel_concept_2021, suliman_establishing_2016}. \newline
\end{itemize}


Différentes manières de se les procurer existent. Pour les médicaments
nécessitant une ordonnance, les étudiants y ont accès grâce à un membre de l'entourage, par détournement de médicaments prescrits, dans l'armoire médicale familiale, illégalement grâce à un dealer ou internet.
Concernant les substances illégales, il y a l'accès par un membre de
l'entourage, l'achat illégal par un dealer ou internet ou bien l'achat dans un pays où la substance est légale \parencite{canterbury_smart_1994} \parencite{c_guillain_les_2021, sharif_assessing_2022}.
\newline

Des aides au sein des différentes écoles existent pour aider les
étudiants dans leur méthode de travail ou pour leur apporter un soutien
psychologique. Par exemple, au sein de l'ULiège, il existe le service de
qualité de vie des étudiants qui peut traiter ce genre de problème.
Cependant, c'est aux étudiants de prendre l'initiative d'aller
discuter de ce qui les préoccupe. Ce n'est pas toujours facile étant donné que c'est un sujet tabou. De plus, différentes mesures ont été proposées par la Chambre des Représentants en 2016 : 
\begin{itemize}

\item
Suivre de près l'évolution de la situation
\item
Organiser des campagnes de sensibilisation ciblant les médecins et pharmaciens
\item
Renforcer la surveillance de vente de stimulants, somnifères et calmants
\item
Lutter contre les ventes illégales de médicaments et rendre les peines plus sévères
\item
Organiser des campagnes  de sensibilisation et prévention auprès des étudiants et de leurs parents \parencite{noauthor_proposition_2016}.

\end{itemize} 

Aux États-Unis, à Oxford, un programme limitant l'accès aux médicaments utilisés pour traiter le TDAH a été mis en place : l'étudiant doit passer par différentes étapes comprenant un appel téléphonique avec le service de counseling des étudiants, assister à un atelier de 90 minutes pour apprendre à améliorer ses techniques d'étude, remplir une feuille d'objectifs de travail suite à cet atelier pendant plusieurs semaines et finalement, assister à un atelier de 1h ayant pour but d'éviter le mésusage et la redistribution de ces médicaments. Lorsque toutes ces étapes sont complétées, l'étudiant peut enfin prendre un rendez-vous avec un médecin. Cependant, aucunes informations ne sont disponibles concernant l'évaluation du programme. 

Il a été prouvé par une étude menée en Australie que les programmes de prévention consistant à informer avaient un impact positif sur les connaissances et attitudes des étudiants face aux psychostimulants. En effet, une augmentation des connaissances liée aux psychostimulants a été observée. Cependant, ce type de prévention utilisée seule n'est pas suffisante pour obtenir une modification du comportement \parencite{vogl_universal_2014}.



\section{Contexte}

Cette consommation de psychotropes peut être expliquée entre autres par le stress occasionné par le système scolaire. Certains auteurs identifient telles causes : la peur de l'échec, le fait de se sentir inadapté au système, des installations d'étude inadéquates, la vulnérabilité financière, la situation familiale, la surcharge de travail, la difficulté des cours, le manque de temps pour étudier, les résultats aux examens et les attentes venant de la famille\parencite{jayasankara_reddy_academic_2018, stankovska_bces_2018, ul_haq_psychometric_2018}. \newline

Concernant la Belgique francophone, selon  Guiot Orlanne, parmi les étudiants questionnés à l'ULB, l'UCL et l'ULiège, plus de la moitié présenterait des symptômes de dépression et d'anxiété depuis la crise sanitaire liée au coronavirus \parencite{orlanne_souffrir_2021}. \newline

En Belgique, les étudiants sont évalués sur leurs connaissances lors de 3 sessions d'examens réparties sur l'année. La première session se passe en janvier et évalue les cours ayant été suivis lors du premier quadrimestre. Celle-ci est précédée d'un blocus de 2 semaines qui permet aux étudiants de finaliser leur étude avant l'évaluation. En juin, les étudiants sont évalués sur les cours suivis au deuxième quadrimestre et cette session est précédée d'une semaine de blocus. Et finalement, la dernière session, aussi appelée "repêchages", se passe d'août à septembre et évalue les cours pour lesquels l'étudiant a obtenu une note inférieure à 10/20 lors des sessions précédentes. Des évaluations certificatives peuvent aussi avoir lieu
pendant l'année pour certains cours, permettant de garantir un travail continu et régulier de la part de l'étudiant. \newline

Cependant, avec ce type de système, les examens sont peu espacés et fort condensés.
Ceux-ci génèrent beaucoup de stress et d'anxiété auprès des étudiants. Les périodes de blocus et d'examens sont intenses : les étudiants doivent retenir des quantités très importantes de matière en une courte durée. De plus, certaines facultés, comme celles de médecine, ingénierie et médecine vétérinaire par exemple, sont compétitives (examens d'entrée, limitation du nombre d'étudiants, ...). Selon un sondage réalisé par la Fédération des étudiants francophones, 68\% des étudiants y ayant répondu ne veulent plus que les examens soient organisés en 3 grosses sessions et 60\% souhaiteraient être évalués tout au long de l'année \parencite{baus_stop_2022}.

En effet, selon Collin, Otero et Monnais, les psychostimulants constituent \guillemotleft de puissants instruments de socialisation, et pour certains de mise en conformité, dans des sociétés caractérisées par un individualisme de masse, où prévaut une exigence d'adaptation permanente à des changements rapides et à une normativité sociale axée sur la responsabilité individuelle, la performance continuelle et la valorisation de l'autonomie \guillemotright \parencite{gagnon_johanne_2008}. 
Le recours aux substances stimulantes représente donc une stratégie d'adaptation au milieu scolaire où la pression sociale est très présente.Les étudiants perçoivent un déséquilibre entre les ressources personnelles et les contraintes imposées par l'environnement académique. Elles leur permettent de concilier travail, étude et vie sociale \parencite{thoer_utiliser_2013}.

\section{Entre croyances et réalité}

Généralement, les étudiants décidant de consommer des substances sont au courant de leurs effets positifs : ce sont les effets recherchés. Cependant, ils ne sont pas toujours au courant des effets négatifs ainsi que des posologies à utiliser pour limiter les conséquences négatives à court et long terme. Souvent, ils pensent que les médicaments sont plus sûrs et moins addictifs que les drogues illicites. Les effets observés par les étudiants peuvent donc être suramplifiés ou être des effets placebo. De plus, ce type de consommation peut avoir un effet boostant sur la confiance en soi et donc donner l'impression d'être plus performant \parencite{c_guillain_les_2021}.
Une étude a démontré que consommer des stimulants ne permettait pas d'augmenter la moyenne scolaire. \newline

Les effets attendus par les étudiants sont l'augmentation d'énergie, rester éveillé plus longtemps, améliorer l'humeur et augmenter la concentration. Pour obtenir ces effets, il est important de respecter une certaine posologie et de bien choisir la substance. Cependant, toute cette énergie et cette motivation ressenties lors de la consommation de certains stimulants ne viennent pas de nulle part. Le corps puise dans ses réserves petit à petit et cela peut mener à l'épuisement. Une surconsommation ou une consommation non adaptée peut entraîner bien des
conséquences \parencite{schifano_benefits_2022}.

\subsection{Les amphétaminiques}

Définition : "Substance excitant le système nerveux central et accroissant les activités psychiques et physiques. Leurs effets secondaires sont fréquents, nombreux et graves (troubles psychiatriques aigus, cardiovasculaires, digestifs)" \parencite{noauthor_amphetaminique_nodate}.

Concernant les amphétaminiques, les risques immédiats sont l'accélération du rythme cardiaque pouvant entraîner une crise cardiaque, une hémorragie cérébrale avec ou sans apoplexie ainsi que l'augmentation de la température corporelle pouvant mener à une surchauffe ou une déshydratation. Ces substances peuvent aussi mener à un état d'épuisement à cause du manque de sommeil et de la sous-alimentation engendrée par l'effet coupe-faim. De plus, à dose élevée, elles peuvent mener à une intoxication qui pourrait provoquer la mort. 

Au niveau des risques à long terme, il y en a deux types : les atteintes psychiques et les atteintes physiques. Parmi les atteintes psychiques, il y a l'anxiété, la dépression, l'épuisement, l'agitation, l'agressivité et la psychose amphétaminique. D'autre part, leur consommation peut mener à des troubles cognitifs tels que des troubles de l'apprentissage, de la concentration ou de mémoire. Ils peuvent aussi augmenter le risque d'apparition de la maladie de Parkinson. Parmi les atteintes physiques, il y a la perte de poids, l'affaiblissement du système immunitaire, la lésion de certains organes ainsi que les \oe dèmes pulmonaires, les troubles du rythme cardiaque, l'assèchement des muqueuses de la bouche et de la gorge, l'épuisement physique ainsi que les troubles du sommeil. De plus, un risque de dépendance élevé existe, même lorsque consommé à petite dose \parencite{koren_use_2021, lappin_psychostimulant_2019, sabbe_use_2022, noauthor_consommation_2022}.

Entre 2000 et 2016, la mortalité associée aux amphétaminiques a augmenté de 160,1\% dans le monde \parencite{mattiuzzi_worldwide_2019}. 

\subsection{Les boissons énergisantes}

Définition : "ces boissons comprennent une variété de boissons non alcoolisées contenant comme ingrédients principaux de la caféine et de la taurine et étant vendues comme des boissons stimulantes améliorant les performances physiques et intellectuelles" \parencite{barrense-dias_les_nodate}.

De nombreuses études ont démontré l'influence néfaste des boissons énergisantes sur la santé. Elles contiennent plusieurs ingrédients problématiques tels que de grandes quantités de sucre qui favorisent l'obésité, l'apparition de diabète et de maladies cardiovasculaires et peuvent nuire à la santé dentaire.

Il y a aussi de la caféine en très grande quantité (aussi présente dans le thé, café, chocolat, etc). En quantité modérée, elle ne présente aucun risque. La quantité recommandée est de maximum 400 mg/jour pour une personne de plus de 13 ans. Néanmoins, la plupart des boissons énergisantes en contiennent une quantité proche ou supérieure à la limite. En addition à d'autres aliments contenant aussi de la caféine, les étudiants se retrouvent facilement à une surconsommation. 

Concernant la taurine, il y en a environ 1000 mg par canette de 250 ml de boisson énergisante. Cela représente une grande quantité étant donné qu'il est recommandé d'en consommer entre 2 et 3 g par jour en complément d'une alimentation équilibrée. Il a été observé sur des animaux que la taurine augmente l'absorption de $Ca^{2+}$ dans le réticulum sarcoplasmique des muscles et qu'il y a une libération accrue de  $Ca^{2+}$, ce qui pourrait avoir comme conséquence d'augmenter la contractilité du myocarde. Cependant, aucune étude réalisée sur les humains n'a réussi à prouver qu'une surconsommation était associée à une toxicité, car il n'existe que des preuves indirectes \parencite{bigard_dangers_2010, dutka_acute_2014, higgins_energy_2018, picard-masson_consumption_2016}.

\subsection{Les benzodiazépines}

Définition : "Les benzodiazépines ont quatre propriétés que l'on retrouve
plus ou moins fortement selon le type de produit : anxiolytique, sédative,
anti-convulsive, hypnotique ou myorelaxante. Ces substances peuvent
notamment être utilisées dans le traitement des troubles bipolaires en phase
maniaque, lors d'un sevrage alcoolique ou encore lors d'une prémédication en
anesthésie" \parencite{allo_barrientos_lenjeu_2019}.

D'après plusieurs études, il existe plusieurs risques associés à la consommation de benzodiazépines. À dose élevée, ces médicaments peuvent donner lieu à des troubles respiratoires, de la confusion/désorientation, de l'amnésie et de la dépression. De plus, ils provoquent des troubles de la mémoire, affectent les performances physiques ainsi qu'intellectuelles et augmentent le risque d'apparition de la maladie d'Alzheimer. Une utilisation prolongée augmente le risque de dépendance et de tolérance à la substance. Par ailleurs, il est dangereux d'associer les benzodiazépines à l'alcool, mais aussi aux opioïdes, car il y a un risque plus élevé de surdose et les benzodiazépines rendent les effets de l'alcool plus intenses. Il existe donc un risque d'arrêt respiratoire \parencite{chen_perceptions_2020, sake_benzodiazepine_2019, noauthor_medicaments_nodate}. \newline

Lorsqu'il y a un excès ou une mauvaise utilisation, l'inverse des effets attendus par les étudiants est observé. Cela est causé par les effets indésirables liés à la consommation d'une trop grosse dose de la substance. On remarque notamment l'apparition de maux de tête, de nervosité, d'irritabilité ainsi que d'insomnie lorsqu'il y a un excès de caféine \parencite{ehlers_risk_2019}.

\section{Conclusion}

Comme décrit ci-dessus, chacune des substances citées ont des effets néfastes sur la santé lors d'une utilisation long terme. Étant donné que la plupart de celles-ci peuvent mener à une addiction, il serait primordial de les utiliser en connaissance de cause et d'avoir un avis médical préalable à toute consommation. Entre 2019 et 2020, il y a eu une augmentation de 48\% de la mortalité associée aux overdoses aux Etats-Unis pour la tranche d'âge 15-24 ans. De plus, il y a eu une augmentation de 150\% de la mortalité liée aux psychostimulants pour la tranche d'âge 20-24 ans \parencite{labossier_stimulant_2022,jones_methamphetamine_2022}. \newline

Beaucoup d'études ont mis en évidence la problématique de consommation de psychostimulants par les étudiants dans le monde entier.
En Belgique, on ne retrouve qu'une seule étude ayant été réalisée en 2018 au sein de 6 universités francophones. Elle a mis en évidence qu'il y avait un mésusage des médicaments stimulants au sein des étudiants francophones dans le but d'améliorer l'étude. En effet, en 2018, un étudiant sur vingt utilisait des stimulants pour des raisons non médicales \parencite{sabbe_use_2022}. \newline

D'autres études réalisées entre 2005 et 2015 en Flandre montraient déjà une augmentation de la consommation de stimulants \parencite{van_wel_changes_2016}.
En 2016, la Chambre des Représentants de Belgique avait donc fait une proposition de résolution reprenant toute une liste de solutions applicables dans toute la Belgique \parencite{noauthor_proposition_2016}.
Malheureusement, les résultats des recherches qui ont suivi sont peu encourageants : les chiffres continuent d'augmenter. \newline

La dernière étude réalisée en Flandre date de 2021. Elle a mis en évidence que les jeunes adultes ayant consommé des médicaments à des fins non médicales, l'avaient fait exceptionnellement et de manière discontinue. Elle a aussi permis de mettre en évidence la facilité avec laquelle il est possible de se procurer ces stimulants au sein de son cercle de connaissances. Cette étude ciblait seulement l'utilisation des médicaments de type analgésiques, stimulants, somnifères et sédatifs sans reprendre les substances illicites \parencite{bawin_federal_nodate}. \newline

Peu d'études ont été réalisées en Belgique et parmi celles-ci, aucune n'a cherché à mettre en évidence les facteurs favorisants, ni la connaissance des risques de consommation des étudiants. Connaitre ces donnée permettrait de mettre en place des actions de prévention adaptées pour essayer de limiter ces risques.

\chapter{Modèle d'analyse proposé}
\section{Question de recherche}

La question de recherche a été formulée comme suit~: \guillemotleft Quels sont les facteurs prédictifs de la consommation de psychostimulants utilisés dans un but de performances scolaires chez les étudiants du cycle supérieur à Liège ? \guillemotright \newline


\section{Objectifs}

\paragraph{Principal :} 

Le but de cette étude est de mettre en évidence les facteurs prédictifs du phénomène de consommation de psychostimulants dans le cycle supérieur à Liège.
\newline

\paragraph{Secondaires : \newline } 
\begin{itemize}

    \item Mesurer l'ampleur du phénomène de consommation de psychostimulants dans le cycle supérieur à Liège.

    \item Évaluer les connaissances des étudiants concernant les risques de consommations des différentes substances.
    
    \item Mettre en place un guide éducatif pour conscientiser les étudiants consommant des psychostimulants.  
    
    \item Essayer de faire réagir les différentes écoles si les résultats sont alarmants en leur faisant parvenir les résultats pour leur donner une idée de la prévalence.

\end{itemize}



\section{Concepts étudiés et hypothèses}

L'hypothèse a été formulée grâce à la méthode hypothético-déductive suite à la lecture d'études réalisées au Canada, USA, et Belgique \parencite{carton_dopage_2018, sabbe_use_2022, noauthor_prevention_2016} : 

\guillemotleft Les facteurs prédictifs théoriques sont :
\begin{itemize}
    \item avoir entre 12 et 25 ans,
    \item être de race blanche,
    \item être en détresse psychologique,
    \item consommer beaucoup d'alcool,
    \item faire partie d'un cercle étudiant / de baptême,
    \item avoir des conflits familiaux,
    \item être un homme,
    \item avoir des notes scolaires faibles,
    \item avoir déjà consommé des substances illicites,
    \item avoir beaucoup d'activités sociales,
    \item sécher des cours.  \guillemotright \newline
\end{itemize}



Le phénomène observé ici est le dopage cognitif des étudiants du cycle supérieur. Les différents concepts sont : \newline

\begin{tabularx}{\textwidth}{| X | X | X |}
    \hline
    \textbf{Concepts} & \textbf{Dimensions} & \textbf{Indicateurs} \\
    \hline \hline
    Dopage cognitif / Consommation de psychostimulants & Performances d'étude & Prévalence de consommation \\
    \hline
    Risques associés à la consommation & Mesure des connaissances & Pourcentage d'étudiants informés sur les risques de consommation \\
    \hline
\end{tabularx}




\chapter{Matériel et méthodes}
\section{Type d'étude}
La méthode choisie est une approche quantitative, celle-ci permettra de
répondre à la question de recherche. La méthode qualitative n'est pas adaptée pour cette recherche, car elle est plus dans la recherche de sens et l’explication d’un phénomène. Étant donné que l'objectif est de quantifier et généraliser les résultats obtenus pour mettre en place un guide préventif, la méthode quantitative est plus adaptée.


Le design choisit est l'étude d'observation transversale rétrospective. Elle permettra de dresser un profil d'étudiant à risque ainsi qu'une liste des substances les plus fréquemment utilisées, d'évaluer leurs habitudes de consommation et permettra de mettre en évidence d'autres données telles que les connaissances concernant les risques de consommation, les raisons de consommation et les périodes les plus propices.

\section{Population étudiée et méthode d'échantillonnage}

\paragraph{Public cible :}les étudiants du cycle supérieur de Liège comprenant ceux de l'Université de Liège ainsi que ceux des hautes écoles de Liège (HEPL, HEL, HELMO et HECH).
  
  Les critères d'inclusion sont définis comme :

\begin{itemize}
\item
  Être un étudiant de l'Université ou d'une Haute école de Liège
\item
  Être inscrit à l'année académique 2022-2023
\item
  Avoir 18 ans ou plus
\end{itemize}

Les critères d'exclusions sont définis comme: 
\begin{itemize}
\item être un étudiant d'une autre ville que Liège
\item ne pas être inscrit dans une école du cycle supérieur
\item ne pas parler français
\item avoir des problèmes cognitifs empêchant de comprendre les questions
\item consommer des substances psychotropes uniquement à visée récréative
\end{itemize}

\paragraph{Échantillonnage :} la méthode d'échantillonnage choisie est le non
probabiliste à participation volontaire. En effet, n'importe quel
étudiant ayant entre 18 et 25 ans et fréquentant une haute école ou
l'université de Liège peut participer à l'étude. Les questionnaires
seront distribués via les différentes plateformes des écoles ainsi que
les différents groupes Facebook des écoles/facultés. Étant un sujet
"tabou", le questionnaire sera anonyme pour attirer un maximum
d'étudiants.
Cette méthode d'échantillonnage est pratique, rapide et sans frais. Cependant, elle donne lieu au biais de sélection : il risque d'y avoir des réponses principalement de personnes pratiquant le dopage cognitif, car elles seront intéressées par le sujet. Cela poserait un problème, car il faut avoir un échantillon aussi homogène que possible et donc avoir des réponses de non-consommateurs. Pour ce faire, il sera important de bien préciser lors de la diffusion du questionnaire que les non-consommateurs doivent aussi répondre.


\paragraph{Calcul de la taille de l'échantillon : }

Aucune recommandations claires n'ont été trouvées dans la littérature scientifique. La formule standard de calcul d'échantillon a dès lors été choisie.

\begin{itemize}
\item
  Taille de la population (N) : 45 000
\item
  Marge d'erreur (e) = 0,05
\item
  Niveau de confiance = 0,95 \newline
  => z = 1,96
\item 
Ecart-type (p) = 0,5 \newline

La formule choisie est donc la suivante : \begin{math}n = \frac{ \frac{z^2 \times p ( 1 - p )} {e^2}}{\frac{z^2 \times p ( 1 - p )} {e^2 \times N}} = 381\end{math} 

\section{Paramètres étudiés et outils de collecte de données}



\section{Contrôle de la qualité}

\section{Composition de l'équipe de recherche}

\section{Aspects règlementaires}

\section{Exploitation des résultats et publications}
  
\textbf{{\hfill\break
  }}
\end{itemize}