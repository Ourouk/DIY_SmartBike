\documentclass[12pt,a4paper]{report}

% We don't use it because xelatex support unicode by default
%\usepackage[utf8]{inputenc}  %mettre latin1 à la place de utf8 si vous utilisez latin1
\usepackage{pdfpages}
\usepackage[T1]{fontenc}
\usepackage[utf8]{inputenc}
\usepackage{microtype}
\usepackage{setspace}
\usepackage[sfdefault]{carlito}
\usepackage{listings}
\usepackage{listings}
\usepackage{xcolor}



% Define YAML style
\lstdefinelanguage{YAML}{
	keywords={true,false,null,y,n},
	keywordstyle=\color{darkgray}\bfseries,
	sensitive=false,
	comment=[l]{\#},
	morecomment=[s]{/*}{*/},
	commentstyle=\color{purple}\ttfamily,
	stringstyle=\color{red}\ttfamily,
	morestring=[b]',
	morestring=[b]"
}
% Set the global style for listings
\lstset{
	basicstyle=\small\ttfamily,
	numbers=left,
	numberstyle=\tiny,
	numbersep=5pt,
	showstringspaces=false,
	breaklines=true,
	frame=tb,
	backgroundcolor=\color{gray!10},
	breakautoindent=true,
	captionpos=b
}

\lstdefinestyle{tree}{
	basicstyle=\ttfamily,
	literate={├}{|}1 {─}{--}1 {│}{|}1 {└}{+}1,
	moredelim=[is][\color{blue}\bfseries]{<}{>}
}


\renewcommand{\baselinestretch}{0}
\usepackage[french, english]{babel}   %langue française et anglais. Si vous utilisez d'autres langues, ajoutez-les ici. Quand vous avez du texte en anglais, il faut ajouter la commande \selectlanguage{english}. Après, rappelez-vous de revenir au français avec \selectlanguage{french}
\usepackage[maxlevel=3]{csquotes}
\usepackage[backend=biber,style=numeric,isbn=false,sorting=none]{biblatex}
%\usepackage{droit-fr}
\DefineBibliographyStrings{french}{in={dans},inseries={dans}}
\addbibresource{bibliographie/Protocole.bib} 


%%defining the new style for legislation
\DeclareBibliographyDriver{legislation}{%
  \printnames{author}%
  \newunit\newblock
  \printfield{title}%
  \newunit
  \printfield{langid}%
  \newunit
  \printfield{shortjournal}%
  \newunit
  \printlist{publisher}%
  \newunit
  \printlist{location}%
  \newunit
  \printfield{year}%
  \newunit
  \printfield{number}%
  \newunit
  \finentry}



\usepackage[left=2cm,right=2cm,top=2cm,bottom=2cm]{geometry} %cette ligne permet de modifié la taille des marges

\usepackage{graphicx}
\usepackage{epigraph}
\setlength\epigraphwidth{13cm}
\usepackage[center,up,labelfont=bf]{caption}
\usepackage{float}
\usepackage{url}
\usepackage{multirow}
\usepackage{tabularx}
\usepackage{hyperref}% this line enable the redirecting from the table of contents
\newcommand{\guil}[1]{\guillemotleft{#1}\guillemotright}    %guillemets 
\newcommand{\guill}[1]{``{#1}''}     %guillements dans les guillemets
\usepackage{gensymb}
% used to remove space in list
\usepackage{enumitem}
\setlist{nosep}



% This permit to modify easily the space on top and under chapter titles
\usepackage{titlesec}{}
\titleformat{\chapter}[display]{\normalfont\huge\bfseries}{\chaptertitlename\ \thechapter}{10pt}{\Huge}
% this alters "before" spacing (the second length argument) to 0
\titlespacing*{\chapter}{0pt}{0pt}{15pt}

%MPut number to the right
\usepackage{titleps}
\renewpagestyle{plain}{%
\sethead{}{}{}
\setfoot[\thepage][][]{}{}{\thepage}
}%
\pagestyle{plain}


\begin{document}

\selectlanguage{french}


\author{Andrea Spelgatti}
\begin{titlepage}
		\begin{flushleft}
			\begin{bfseries}
				\includegraphics[scale = 1]{media/images/HEPL.PNG}
			\end{bfseries}
		\end{flushleft}
		\hspace{2cm}
		\begin{center}
			\begin{center}
				\begin{Large}
					Projet \\[1cm]
					\textbf{IOT}\\
					Pierre De Fooz\\[1.5cm]
				\end{Large}
			\end{center}
			\rule{\linewidth}{0.5mm}\\[0.5cm]
			{\huge\bfseries Projet IOT : Smart Bike}
			\rule{\linewidth}{0.5mm}\\[2cm]
			\includegraphics[scale = 0.2]{media/images/velo}
		\end{center}
		\vspace{2 cm}
		\begin{minipage}{0.75\textwidth}
			\begin{flushleft}
				SPELGATTI Andrea \\
				Alexandre Gallez \\
				Master Ingénieur Industriel | M18\\
				Année 2022-2023
			\end{flushleft}
		\end{minipage}
		\vspace{3cm}
		\begin{center}
			{\bfseries \today}
		\end{center}
\end{titlepage}
%
\null
%\newpage %used if having a page de garde

%\pagenumbering{roman} % numérotation en chiffres romains
%\setcounter{page}{1}

%\addcontentsline{toc}{chapter}{Résumé}
\chapter*{Résumé} 


Résumé ici

\begin{singlespace}
\textbf{Mots clés~:} Mots-clés ici
\end{singlespace}
%\addcontentsline{toc}{chapter}{Abstract}
\chapter*{Abstract}
\selectlanguage{english}
Abstract

\begin{singlespace}
\textbf {Keywords ~:} keywords
\end{singlespace}
\selectlanguage{french}
%\input{titres/remerciements.tex}

\begin{singlespace}
    \selectlanguage{french}
    \tableofcontents % table des matières
    %\listoffigures % si vous avez des images... si vous n'en avez pas, effacez cette ligne
\end{singlespace}

% fin numérotation en chiffres romains
% début numérotation en chiffres arabs3
\pagenumbering{arabic}
\setcounter{page}{0}


%dans les fichier vous trouverez des exemples d'usage des différentes commandes de LaTeX
\begin{onehalfspace}
\thispagestyle{empty}
	\chapter*{Introduction}
\section*{Le Prototype Révolutionnaire de Vélo Intelligent}

Dans le vaste panorama de l'innovation technologique, émerge une étoile qui brille pour tous. Notre projet, porteur de changement, aspire à réaliser un rêve partagé : rendre les avantages des vélos intelligents modernes accessibles à chacun, quel que soit le portefeuille. Le prototype que nous présentons aujourd'hui s'épanouit pour concrétiser cette vision inclusive.

À travers cette initiative, nous aspirons à donner à tous les passionnés de la mobilité à deux roues les opportunités et les prouesses de l'univers des vélos intelligents, sans égard aux contraintes financières. Ce projet englobe une gamme d'innovations qui repoussent les frontières de la sécurité, de la praticité et de la connectivité pour tous les cyclistes.

Examinons de plus près ce prototype qui brille par ses caractéristiques exceptionnelles :

\begin{itemize}
	\item \textbf{Détection d'Accident avec Caméra de Bord:} Une caractéristique qui va au-delà de la simple sécurité en offrant une couche supplémentaire de protection. Non seulement ce vélo peut détecter les accidents potentiels, mais il peut également enregistrer ces moments cruciaux grâce à une caméra embarquée .
	\item \textbf{Déverrouillage du Vélo via RFID:} La commodité est à l'honneur avec cette fonctionnalité. Grâce à la technologie RFID, il vous suffit d'approcher votre vélo pour qu'il se déverrouille automatiquement, éliminant ainsi le besoin de clés physiques pour garentir une sécurité maximal.
	\item \textbf{Détection de Vol:} Pour dissuader les voleurs et protéger votre investissement, ce vélo intelligent est équipé d'un système de détection de vol. En cas d'une tentative d'ouverture forcée, le vélo saura réagir.
	\item \textbf{Traceur GPS:} La perte de votre vélo ne sera plus source d'inquiétude grâce au traceur GPS intégré. Même lorsque vous n'êtes pas à proximité, vous pourrez localiser votre vélo sur une carte en temps réel.
	\item \textbf{Éclairage Automatique:} Adaptabilité est le maître-mot avec cet éclairage automatique. Le capteur de lumière ambiante ajuste automatiquement l'éclairage du vélo en fonction des conditions extérieures, garantissant une visibilité optimale à tout moment.
	\item \textbf{Sauvegarde sur un Cloud Dédié:} Toutes les informations pertinentes liées à votre vélo seront sauvegardées de manière sécurisée sur un cloud spécialement dédié, garantissant que vous ne perdrez jamais de données cruciales.
	\item \textbf{Tableau de Bord Android:} Pour une expérience utilisateur enrichissante, ce vélo intelligent est livré avec un tableau de bord fournissant une mine d'informations utiles à portée de main.
\end{itemize}

En résumé, ce prototype de vélo intelligent incarne une démarche qui met la technologie au service de tous. Il ouvre la voie à une ère où les avantages des vélos intelligents ne sont plus réservés à quelques privilégiés, mais deviennent une réalité tangible pour tous les amateurs de deux-roues. Restez à l'écoute pour découvrir comment ce projet ambitieux pourrait redéfinir notre façon de percevoir la mobilité urbaine.
\subsection{l architecture dossier}

Git s'est imposé comme l'outil clé favorisant le travail d'équipe, l'organisation du code et la traçabilité des modifications dans notre projet de vélo intelligent. Toutefois, pour des raisons structurelles du système, certains éléments nécessaires ne sont pas inclus, car ils doivent résider à des emplacements spécifiques. Cette approche responsable vise à garantir la sécurité des données tout en tirant parti des avantages de Git pour l'innovation collaborative.

en dossier principal, on retrouve:

le dossier esp32 qui regroupe tous les codes de test Arduino et un dossier final qui mélange le code créé dans les autres dossiers ( finalesp32 ). 

dans le dossier esp8266, on retrouve le code python de test et le code final en Arduino.

dans le dossier sensor exemple, on retrouve tous les codes de tests des composants qui ont été la base du développement des codes finaux. 

le dossier rapport contient l'architecture Latex de ce rapport. 

le dossier RaspberryPi contient le code de node red , le code de chaque sous script exécuté par node red. 

dans le dossier rocky-linux on y retrouve tous les différents dockers et leur installation. 


\noindent
voici le résultat de la commande tree -d -a de notre archive :
\begin{lstlisting}[style=tree]
 
	├── .git
	│   ├── branches
	│   ├── hooks
	│   ├── info
	│   ├── logs
	│   ├── objects
	│   │   ├── info
	│   │   └── pack
	│   └── refs
	│       ├── heads
	│       ├── remotes
	│       │   └── origin
	│       └── tags
	├── esp32
	│   ├── accelero
	│   ├── accelerometre
	│   │   └── main
	│   ├── final
	│   │   ├── finalesp32
	│   │   └── main
	│   ├── gps
	│   │   └── m
	│   ├── lora
	│   │   └── main
	│   └── temp and humidity detector
	│       └── main
	│           └── main
	├── esp8266
	│   └── arduino
	│       └── esp8266
	├── Rapport
	│   ├── bibliographie
	│   ├── media
	│   │   └── images
	│   ├── textes
	│   └── titres
	├── RaspberryPi
	│   └── usages
	│       ├── bluetooth
	│       │   └── node
	│       ├── nodered
	│       └── scripts
	│           ├── 6axis
	│           │   └── __pycache__
	│           ├── lora
	│           └── __pycache__
	├── rocky-linux
	│   └── docker
	│       ├── eclipse-mosquitto
	│       └── node-red
	└── sensors-examples
	├── grove_6_axis
	│   ├── LSM6DS3-for-Raspberry-Pi
	│   └── __pycache__
	└── __pycache__
\end{lstlisting}

\section{schémat}

%TODO  mettre le schéma d architecture 


   	\chapter{Hardware}
\section{Raspberry Pi}
L'intégration d'une Raspberry Pi dans l'écosystème de l'Internet des Objets (IoT) est une démarche pragmatique et fonctionnelle. Avec la version récente de Raspbian, baptisée Bullseye, l'accent est mis sur la stabilité et la compatibilité, ce qui en fait un choix solide pour les projets IoT. En utilisant cette configuration, il est possible de mettre en place des dispositifs qui collectent des données provenant de capteurs divers et les transmettent à des systèmes centraux pour analyse.
Nous combinons les capacité de la rpi avec Node-red,  en tant qu'outil de programmation visuelle, facilite la création de flux de données et d'automatisations en utilisant des nœuds interconnectés (nous y revienderons plus tard).
\subsection{GrovePi}
GrovePi, une extension matérielle conçue pour la Raspberry Pi, constitue une solution intéressante pour étendre les capacités de la carte dans le domaine de l'Internet des Objets (IoT). En intégrant des connecteurs Grove standardisés, GrovePi simplifie la connexion de divers capteurs et actionneurs sans nécessiter de soudure ou de câblage complexe.

Nous avons utilisé les bibliothèques standarts fournies par le projets, il est à noté que nous avons du faire de petite modifications, pour rendre compatible grovePI et Bulleyes.
\subsection{LORA HAT}
\section{ESP32}
L'ESP32 intègre un processeur à deux cœurs, une connectivité Wi-Fi et Bluetooth, ainsi qu'une variété d'interfaces périphériques. Grâce à sa puissance de traitement, il peut exécuter des tâches complexes et prendre en charge des fonctionnalités avancées. En tant que plate-forme de développement, l'ESP32 est pris en charge par diverses bibliothèques et outils de programmation, ce qui en fait un choix polyvalent pour les développeurs.

Il recoit des données du GPS (via les gpip)et les envoies via LORA à notre RPI.
\section{ESP8266}
Ce module est doté d'un microcontrôleur intégré avec une connectivité Wi-Fi intégrée. Bien que moins puissant que l'ESP32, l'ESP8266 reste tout à fait capable d'exécuter des tâches IoT courantes telles que la collecte et la transmission de données. Son faible coût et sa faible consommation d'énergie en font un choix populaire pour les projets nécessitant une connectivité Wi-Fi.
Dans le cadre de notre projet sont utilisations reste très basique, mais nous avons eu l'occasion d'entrevoir son grand potentiel.  Simple activation de led.


\section{estimation de prix}

\begin{itemize}
	\item raspberri pi 
	
	€150
	\item LORA hat
	
	€17 - €42
	\item Pi cam
	
	€9 - €25
	\item Light sensor
	
	€1.70 - €8.50
	\item RFID
	
	€4 - €17
	\item Button
	
	€0.85 - €4.20
	\item LCD screen
	
	€8.50 - €25.50
	\item LED
	
	€0.08 - €0.85
	\item Buzzer
	
	€0.85 - €4.20
	\item Relay
	
	€1.70 - €8.50
	\item Gyroscope
	
	€4 - €17
	
	\item ESP32	
	€5 - €15
	
	\item gps
	
	€10 - €40
	
	\item ESP8266	
	€5 - €10
	
	\item température sensor	
 
	€2 - €10
\end{itemize}


\chapter{Le marché }

le prix total du produit se trouve  €170 - €380.Il faut bien se rendre compte que le produit est une version de test pour utiliser le maximum de technologie IOT. Dans une application plus réelle de notre solution. nous utiliserons une seule carte pour centraliser les mesures. Donc il est facilement réalisable d'économiser une 100 ene d'euros.et en choisissant de bons composants pas trop chère en grande quantité le cout final du produit ne sera pas plus de 70 euros. Il faut noter que pour réaliser un tel gain, il est nécessaire de recourir à l'ingénierie électronique en fabriquant une carte mère dédier. en fixant cet objectif le produit pourrait se vendre sans trop de problème pour €200 tout en pouvant dégager une marche agréable afin de rembourser le R \& D , le marketing et les diverses charges…  


\section{Étude de marché }


Segmentation de la Clientèle : Notre produit cible principalement les centres urbains où la demande de solutions de mobilité alternatives est élevée. Les jeunes professionnels, les étudiants et les citadins soucieux de l'environnement sont les segments clés visés.


Avantages Concurrentiels : En offrant des fonctionnalités telles que la détection d'accident, la sécurité renforcée et la connectivité GPS à un prix abordable, notre vélo intelligent se démarque de la concurrence. Cela attire des clients cherchant un équilibre entre fonctionnalités haut de gamme et budget.


Potentiel de Croissance : Avec des prévisions indiquant une augmentation continue de la demande de vélos intelligents et d'accessoires, notre produit abordable a le potentiel de capturer une part substantielle de ce marché en pleine croissance.

Notre proposition de vélo intelligent abordable trouve sa place dans l'évolution constante du cyclisme en Belgique. Au cours des trois dernières années, une moyenne de 580 000 vélos a été vendue, dont 2/5 sont électriques, témoignant d'un intérêt grandissant pour des solutions de mobilité modernes. Avec une pénétration du marché estimée à 1\%, cela ouvre la voie à la vente de 5 800 appareils par an uniquement en Belgique. Évalué à 200 euros par unité, ce potentiel se traduit par un chiffre d'affaires de 1 million d'euros par ans. En ciblant initialement le marché belge, notre produit aspire à capitaliser sur cette opportunité naissante et à répondre à la demande croissante en matière de vélos intelligents abordables.


   	\chapter{Layer 2 - Interface Utilitaire}
   	\chapter{sécurité}
\section{mot de passe}

 nous avons pris des mesures rigoureuses. Les mots de passe ont été stratégiquement appliqués à divers niveaux de notre système, notamment sur les connexions SSH, Node-RED et MongoDB. Cette approche multicouche renforce la confidentialité des données, prévenant tout accès non autorisé et assurant une expérience utilisateur sûre et protégée à chaque étape.
 
\section{Sécuriser les communications}

nous avons adopté une approche intégrée. Les communications au sein de notre système sont désormais protégées par des certificats de sécurité pour MQTT, MongoDB et Node-RED. Ces certificats agissent comme des boucliers numériques, garantissant la confidentialité des échanges et prévenant toute intrusion non autorisée


\section{API et applications externes utilisées}

chaque requête API est étroitement surveillée et protégée par des tokens de connexion. Ces tokens agissent comme des gardiens numériques, permettant uniquement l'accès autorisé aux informations sensibles. Cette stratégie de sécurité robuste garantit que seules les interactions légitimes ont accès.

\section{Docker}

 nous avons opté pour Docker en tant qu'outil de protection supplémentaire. Les conteneurs Docker offrent une isolation rigoureuse pour chaque application et service, empêchant la propagation de vulnérabilités potentielles.
   	\chapter{Interaction avec des services en ligne}
\end{onehalfspace}


% Here you should put bibliography related content
\printbibliography



\end{document} 